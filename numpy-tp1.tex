\documentclass[12pt]{article}
\usepackage[utf8]{inputenc}
\usepackage[T1]{fontenc}
\usepackage{titling}
\usepackage[lmargin=2cm,rmargin=2cm,bottom=5em,top=5em]{geometry}
\usepackage{mhchem}
\usepackage{enumitem}
\usepackage[
	printsolution=false,
	exercisename=Exercice,
	exercisenameposition=inline,
	exercisespaceabove=1ex,
	exercisespacebelow=.5ex
	]{exercises}
\usepackage{lastpage}
%\usepackage[lastexercise]{exercise}


% Default fixed font does not support bold face
\DeclareFixedFont{\ttb}{T1}{txtt}{bx}{n}{12} % for bold
\DeclareFixedFont{\ttm}{T1}{txtt}{m}{n}{12}  % for normal

% Custom colors
\usepackage{color}
\definecolor{deepblue}{rgb}{0,0,0.5}
\definecolor{deepred}{rgb}{0.6,0,0}
\definecolor{deepgreen}{rgb}{0,0.5,0}

\usepackage{listings}

% Python style for highlighting
\newcommand\pythonstyle{\lstset{
language=Python,
basicstyle=\small\ttm,
otherkeywords={self},             % Add keywords here
keywordstyle=\ttb\color{deepblue},
emph={MyClass,__init__},          % Custom highlighting
emphstyle=\ttb\color{deepred},    % Custom highlighting style
stringstyle=\color{deepgreen},
frame=tb,                         % Any extra options here
showstringspaces=false            % 
}}


% Python environment
\lstnewenvironment{python}[1][]
{
\pythonstyle
\lstset{#1}
}
{}

% Python for external files
\newcommand\pythonexternal[2][]{{
\pythonstyle
\lstinputlisting[#1]{#2}}}

% Python for inline
\newcommand\pythoninline[1]{{\pythonstyle\lstinline!#1!}}

\newcommand{\HRule}[1]{\rule{\linewidth}{#1}}

\usepackage{fancyhdr}

\usepackage{amsthm}
 
\makeatletter
\newtheoremstyle{HintStyle}
  {3pt}% space before
  {0pt}% space after
  {\em\small
  %\addtolength{\@totalleftmargin}{-0.5em}
   %\addtolength{\linewidth}{-0.1em}
   \parshape 1 0.025\linewidth 0.95\linewidth}% body font
  {}% indent
  {\bfseries}% header font
  {:}% punctuation
  {\newline}% after theorem header
  {}% header specification (empty for default)
\makeatother

\theoremstyle{HintStyle}
\newtheorem*{remark}{Indications}

\makeatletter
\newtheoremstyle{ExampleStyle}
  {3pt}% space before
  {3pt}% space after
  {\addtolength{\@totalleftmargin}{3.0em}
   \addtolength{\linewidth}{-3.0em}
   \parshape 1 3.5em \linewidth}% body font
  {}% indent
  {\bfseries\em}% header font
  {:}% punctuation
  {\newline}% after theorem header
  {}% header specification (empty for default)
\makeatother

\theoremstyle{ExampleStyle}
\newtheorem*{ex}{Exemple(s)}

\makeatletter
\def\enumfix{%
\if@inlabel
 \noindent \par\nobreak\vskip-\parskip\vskip-\baselineskip\hrule\@height\z@
\fi}

\let\oldenumerate\enumerate
\def\enumerate{\enumfix\oldenumerate}

\pagestyle{empty}
%\fancyhf{}
%\cfoot{\thepage}

%\lhead{COUCOU}
%\rhead{
%text\\
%text\\
%text\\
%text}

   
\fancypagestyle{myfancy}
{
    \fancyhf{}
    \renewcommand{\headrulewidth}{0pt}
    \renewcommand{\footrulewidth}{0.1pt}
    %\lfoot{Some Text}
    \rfoot{\thepage/\pageref{LastPage}}
}
\fancypagestyle{page1}
{
   \fancyhf{}
   \renewcommand{\headrulewidth}{1pt}
   %\rule{\linewidth}{\headrulewidth}\\[3pt]% <- added
   
   \lhead{Université d'Adrar\\Département d'informatique et mathématiques}
   %\chead{~\\even more text}
   \rhead{Outils de programmation pour les math.\\$1^\text{ère}$ année MI}
   \rfoot{\thepage/\pageref{LastPage}}
}

\title{\vspace{-.7ex} \large \bfseries{TP1:} Introduction au langage de programmation Python
}
\setlength{\parindent}{0cm}
\date{}
\predate{}
\postdate{}
\setlength{\droptitle}{-3em} 
%\posttitle{\par\end{center}}
\begin{document}
%\setlength{\headheight}{80pt}

\pagestyle{myfancy}
\maketitle
\thispagestyle{page1}
\vspace{-5em}

\begin{exercise}
	Ecrire un programme qui teste la parité d'un nombre naturel.	
\end{exercise}

\begin{exercise}
	Ecrire un programme qui affiche la valeur absolue d'un entier donné.	
\end{exercise}

\begin{exercise}
Ecrire un programme qui lit deux réels $x$, $y$ et qui:
\begin{enumerate}
 \item Affiche la surface du cercle dont le rayon est $x$ ($\pi x^{2}$).
 \item Affiche la surface d'un rectangle de dimensions $x$ et $y$ ($x\times y$
  ).
  \item Affiche la surface du cylindre dont le rayon est $x$ et la 
  hauteur est $y$ ($2\pi x\left(x+y\right)$).
  \end{enumerate}
    \begin{remark}
  \begin{enumerate}
  \item Utiliser le package \pythoninline{math} pour avoir la constante $\pi$:\\
  \pythoninline { import math}
  \item Utiliser le help sur le module \pythoninline{math}.
  \end{enumerate}
  \end{remark}
 \end{exercise}

\begin{exercise}
	Ecrire un programme en lui fournissant le moi présent, donne le moi suivant.
	On vous demande de donner deux solutions: une en utilisant 'if' et l'autre sans 'if'.	
	\begin{remark}
	Utiliser la méthode \pythoninline{index} du type liste pour avoir l'index du moi dans une liste des chînes de caractères. 
	\end{remark}
\end{exercise}

\begin{exercise}
	Ecrire un programme qui fait la conversion des coordonnées cartisiennes aux coordonnées polaires.
\end{exercise}

\begin{exercise}
\begin{enumerate}
	\item La fonction \pythoninline{time} du package \pythoninline{time} donne le nombre de secondes écoulées depuis le $1$er janvier $1970$ 00:00:00 UTC.
	Transformer ce nombre de secondes en nombre d'années, de jours, d'heures, de minnutes et de secondes (en prenant simplement $1$ an= $365$ jours).
	\item On utilise souvent cette fonction pour compter le temps écoulé dans un calcul donné. 
	Utiliser la fonction \pythoninline{sleep} du même package pour simuler une longue opération, et compter le temps. Que remarquez-vous?
\end{enumerate}
\end{exercise}

\begin{exercise}
	Ecrire un programme qui clacule la somme des chiffres composant la base à la puissance spécifiée.
	
	\begin{ex}
	\begin{enumerate}
	\item base=$2$; puissance=$10$; résultat: $7$ ($=1+0+2+4$)
	\item base=$3$; puissance=$6$; résultat: $18$ ($=7+2+9$)
\end{enumerate}
	\end{ex}
	\begin{remark}
	On vous recommende d'utilser les fonctions \pythoninline{sum} et \pythoninline{map} qui opèrent sur les lsites. Utliser le help pour en avoir une idée.
	\end{remark}
\end{exercise}

\begin{exercise}
	Ecrire un programme qui résout une équation quadratique $ax^2+bx+c=0$ en donnant en entrée les trois coefficients $a$, $b$, $c$.
	
  \begin{remark}
  Utiliser le package \pythoninline{cmath} pour effectuer des calculs dans le domaine complex\\
  \pythoninline{import cmath}
  
  \end{remark}
\end{exercise}
%\begin{solution}[3cm]
	%The result is $1 + 1 = 2$
%\end{solution}


\end{document}
