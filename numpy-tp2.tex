\documentclass[12pt]{article}
\usepackage[utf8]{inputenc}
\usepackage[T1]{fontenc}
\usepackage{titling}

\usepackage{hyperref}
\usepackage{interval}
\usepackage[lmargin=2cm,rmargin=2cm,bottom=5em,top=5em]{geometry}
\usepackage{mhchem}
\usepackage{enumitem}
\usepackage[
	printsolution=false,
	exercisename=Exercice,
	exercisenameposition=inline,
	exercisespaceabove=1ex,
	exercisespacebelow=.5ex
	]{exercises}
\usepackage{lastpage}
%\usepackage[lastexercise]{exercise}


% Default fixed font does not support bold face
\DeclareFixedFont{\ttb}{T1}{txtt}{bx}{n}{12} % for bold
\DeclareFixedFont{\ttm}{T1}{txtt}{m}{n}{12}  % for normal

% Custom colors
\usepackage{color}
\definecolor{deepblue}{rgb}{0,0,0.5}
\definecolor{deepred}{rgb}{0.6,0,0}
\definecolor{deepgreen}{rgb}{0,0.5,0}

\usepackage{listings}

% Python style for highlighting
\newcommand\pythonstyle{\lstset{
language=Python,
basicstyle=\small\ttm,
otherkeywords={self},             % Add keywords here
keywordstyle=\ttb\color{deepblue},
emph={MyClass,__init__},          % Custom highlighting
emphstyle=\ttb\color{deepred},    % Custom highlighting style
stringstyle=\color{deepgreen},
frame=tb,                         % Any extra options here
%breaklines=true,
showstringspaces=false            %
}}


% Python environment
\lstnewenvironment{python}[1][]
{
\pythonstyle
\lstset{#1}
}
{}

% Python for external files
\newcommand\pythonexternal[2][]{{
\pythonstyle
\lstinputlisting[#1]{#2}}}

% Python for inline
\newcommand\pythoninline[1]{{\pythonstyle\lstinline!#1!}}

\newcommand{\HRule}[1]{\rule{\linewidth}{#1}}

\usepackage{fancyhdr}

\usepackage{amsthm}
 
\makeatletter
\newtheoremstyle{HintStyle}
  {3pt}% space before
  {0pt}% space after
  {\em\small
  %\addtolength{\@totalleftmargin}{-0.5em}
   %\addtolength{\linewidth}{-0.1em}
   \parshape 1 0.025\linewidth 0.95\linewidth}% body font
  {}% indent
  {\bfseries}% header font
  {:}% punctuation
  { }% after theorem header
  {}% header specification (empty for default)
\makeatother

\theoremstyle{HintStyle}
\newtheorem*{remark}{Indications}

\makeatletter
\newtheoremstyle{ExampleStyle}
  {3pt}% space before
  {3pt}% space after
  {\addtolength{\@totalleftmargin}{3.0em}
   \addtolength{\linewidth}{-3.0em}
   \parshape 1 3.5em \linewidth}% body font
  {}% indent
  {\bfseries\em}% header font
  {:}% punctuation
  {\newline}% after theorem header
  {}% header specification (empty for default)
\makeatother

\theoremstyle{ExampleStyle}
\newtheorem*{ex}{Exemple(s)}

\makeatletter
\def\enumfix{%
\if@inlabel
 \noindent \par\nobreak\vskip-\parskip\vskip-\baselineskip\hrule\@height\z@
\fi}

\let\oldenumerate\enumerate
\def\enumerate{\enumfix\oldenumerate}

\pagestyle{empty}
%\fancyhf{}
%\cfoot{\thepage}

%\lhead{COUCOU}
%\rhead{
%text\\
%text\\
%text\\
%text}

   
\fancypagestyle{myfancy}
{
    \fancyhf{}
    \renewcommand{\headrulewidth}{0pt}
    \renewcommand{\footrulewidth}{0.1pt}
    %\lfoot{Some Text}
    \rfoot{\thepage/\pageref{LastPage}}
}
\fancypagestyle{page1}
{
   \fancyhf{}
   \renewcommand{\headrulewidth}{1pt}
   %\rule{\linewidth}{\headrulewidth}\\[3pt]% <- added
   
   \lhead{Université d'Adrar\\Département d'informatique et mathématiques}
   %\chead{~\\even more text}
   \rhead{Outils de programmation pour les math.\\$1^\text{ère}$ année MI}
   \rfoot{\thepage/\pageref{LastPage}}
}

\title{\vspace{-.7ex} \large \bfseries{TP2:} Boucles, fonctions, \& vecteurs
}
\setlength{\parindent}{0cm}
\date{}
\predate{}
\postdate{}
\setlength{\droptitle}{-4em} 
%\posttitle{\par\end{center}}
\begin{document}
%\setlength{\headheight}{80pt}

\pagestyle{myfancy}
\maketitle
\thispagestyle{page1}
\vspace{-6em}
Utiliser le fragment suivant pour générer les données sur lesquelles on va travailler.


\begin{python}
	import numpy as np  
	np.random.seed(1234)
	d=np.random.rand(10)
	e=np.random.randint(1,20,10)
\end{python}
On vous recommende d'utiliser le help pour examiner les fonctions utilisées dans ce fragment. i.e. 
 \pythoninline{np.random.seed}, \pythoninline{np.random.rand}, \pythoninline{np.random.randint}.
%On utilise \pythoninline{seed} pour que tous les ètudiants auront les même données. 

\begin{exercise}
%La dérivée d'une fonction $\operatorname{f}$ à un point $x$ peut être  approximée par la fonction:
%$$\operatorname{g} (x,\delta)=\frac{\operatorname{f}(x+\delta)-\operatorname{f}(x)}{\delta}$$
%Ecrire une fonction qui accepte comme paramètres: une fonction d'une seule variable $\operatorname{f}(x)$, un point $x$, et 
\begin{enumerate}\item Ecrire une fonction qui accepte en entrée un vecteur $v$ et qui retourne la moyenne $\bar{v}$ et l'écart-type $\sigma_v$ du vecteur $v$. Où:

\begin{minipage}{.4\linewidth}
$$\bar{v}=\frac{\sum_{i=0}^n{v_i}}{n}$$
\end{minipage}
et
\begin{minipage}{.4\linewidth}
$$\sigma_v=\sqrt{\frac{\sum_{i=0}^n{(v_i-\bar{v})^2}}{n}}$$
\end{minipage}

\item Ecrire une autre fonction qui fait le même calcul en  un seul passage sur les données. 
i.e. On calcul la moyenne et l'ecart-type dans une seule itération. 
\begin{remark}
Calculer la somme et la somme des carrées durant le seul passage sur les données. 
\end{remark}
\item Appliquer ces  fonctions aux vecteurs $d$ et $e$.  Donnent-elles le même résultat?
\item Dans l'analyse de données on a souvent besoin d'effectuer des transformations sur les données. Voici quelques exemples: 
\begin{itemize}
	\item [Standardisation] Ecrire une fonction qui effectue la soustraction de la moyenne et divise par l'ecart type et retourne le vecteur qui en résulte.
	Appliquer cette fonction sur le vecteur $d$ et donner le nombre des nombres négatifs dans le vecteur retourné.
	\item [Normalisation] Ecrire une focntion qui effectue la normalisation d'un vecteur passé comme paramètre. 
	La fonction doit aussi accepter un paramètre qui permet de choisir entre norme 1 ($\ell_1-$norm)\footnote{\url{http://mathworld.wolfram.com/L1-Norm.html}} 
	ou bien norme 2 ($\ell_2-$norm)\footnote{\url{http://mathworld.wolfram.com/L2-Norm.html}}  (Chercher la définition de ces deux termes sur Internet).
	\item [Mise à l'echelle] Ecrire une fonction qui transforme les données dans un vecteur, à l'interval  $\interval{0}{1}$. 
	Pour chaque  élément dans le vecteur en entrée, on enlève la valeur minimale et on divise par l'étendue.
	On aura toujours au moins un $0$ et un $1$ dans le vecteur résultat.
\end{itemize}
\item Appliquer touts les traitements précédents (Standardisation, normalisation,\dots) en série sur un vecteur donné. 
i.e. On applique la standardisation, ensuite on applique la normalisation au résultat de la standardisation,\dots etc.
Procéder en définissant une liste de fonctions et en utilisant la fonction \pythoninline{reduce}.
\end{enumerate}
\end{exercise}

\begin{exercise}
	Ecrire une fonction qui retourne la médiane du vecteur $v$. i.e. La valeur qui permet de couper l'ensemble des valeurs en deux parties égales.
	Appliquer-la sur $d$ et $e$.
\begin{remark}
On doit d'abord trier le vecteur (en utilisant la fonction \pythoninline{sort}). 
Si le nombre des éléments du vecteur est impair, la médiane est au milieu du vecteur trié.
Sinon, la médiane est la moyenne des deux éléments au milieu.

\end{remark}


\end{exercise}


\begin{exercise}
Dans cet exercice, on considère uniquement le vecteur $e$ qui contient des valeurs entières.

Créer deux vecteurs de même taille résumant les données dans $e$:
\begin{itemize}
\item Le premier contient les occurrences uniques des valeurs apparaissant dans $e$.
\item Le deuxième contient le nombre d'occurences (i.e. la fréquence) de ces valeurs unqiues dans $e$.
\end{itemize}

\begin{remark}
Utiliser la fonction \pythoninline{unique}
\end{remark}
\begin{enumerate}
 \item Exprimer les occurrences sous forme de probabilités.
 \item Quelle est la valeur la plus fréquente dans le vecteur?
 \item Calculer la somme, la moyenne et l'ecart type en utilisant ces deux vecteurs.
 \item Utiliser la recherche dichotomique pour trouver si un nombre généré aléatoirement existe dans le vecteur.
\begin{remark}
Utiliser la fonction \pythoninline{searchsorted} pour une recherche dichotomique dans un vecteur trié.
Utliser une des fonctions mentionnées dans le fragment ci-dessus pour générer le nombre aléatoir.
\end{remark}

 \item Calculer la médiane en utilisant uniquement les deux vecteurs.
 \begin{remark}
Utiliser la fonction \pythoninline{cumsum}.
\end{remark}

 \item regénérer les valeurs du vecteur original trié.
 \begin{remark}
Utiliser la fonction \pythoninline{repeat}.
\end{remark}

 \item Générer un echantillon sans remise de taille $5$ du vecteur $e$, 
 \begin{enumerate}[label=\roman*]
	\item En considérant seulement les valeurs uniques.
	\item En tenant compte des répétitions.
 \end{enumerate}
 
 \begin{remark}
 Utiliser la fonction \pythoninline{random.choice} 
 \end{remark}
 \item Afficher les éléments pairs du vecteur $e$. Afficher leur nombre.
 \item Afficher les éléments entre $10$ et $20$.
 
 \end{enumerate}
    %\begin{remark}
  %\begin{enumerate}
  %\item Utiliser le package \pythoninline{math} pour avoir la constante $\pi$:\\
  %\pythoninline { import math}
  %\item Utiliser le help sur le module \pythoninline{math}.
  %\end{enumerate}
  %\end{remark}
 \end{exercise}

\begin{exercise}
	Ecrire une fonction qui re\c coit un vecteur $v$ et une valeur $x$ comme paramètres. 
	Elle doit retourner la valeur dans $v$ qui est la plus prôche de $x$.
	Appliquer-la au vecteur $d$ et une valeur  aléatoirement générée.
	  
\end{exercise}

\begin{exercise}
	Ecrire une fonction qui calcule la moyenne mobile d'un vecteur $v$ sur une fenêtre $n$. 
	i.e. On doit caculer un nouveau vecteur qui contient les moyennes de chaque $n$ éléments consecutifs du vecteur $v$.
\end{exercise}

%~ \begin{exercise}

%~ \end{exercise}

%~ \begin{exercise}
	%~ \begin{ex}
	
	%~ \end{ex}
	%~ \begin{remark}
	
	%~ \end{remark}
%~ \end{exercise}

%\begin{solution}[3cm]
	%The result is $1 + 1 = 2$
%\end{solution}


\end{document}
